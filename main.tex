\documentclass[11pt]{article}

% Required packages
\usepackage{bm,bbm}
\usepackage{amsmath,amssymb,amsthm,cancel}
\usepackage{fullpage, algorithm, algpseudocode}
\usepackage{minted, caption}

% Color references
\usepackage[
    colorlinks=true, citecolor=blue, linkcolor=blue]{hyperref}

% More shortcuts
\newcommand{\ve}{\varepsilon}
\newcommand{\dl}{\delta}

%\everymath{\displaystyle}

\newcommand{\homework}[5]{
	\noindent
    \begin{center}
    	\framebox{
        	\vbox{
            	% Course Title and Date
            	\hbox to \hsize { \textsc{ #5 } \hfill #1 }
                \vspace{4mm}
                % Title of handout/homework
                \hbox to \hsize { {\Large \hfill \textsc{#2} \hfill} }
                \vspace{2mm}
                % Name, Cornell NetID
                \hbox to \hsize {\hfill \textbf{\textsf{#3}} - \texttt{#4@cornell.edu} \hfill }
                \vspace{1mm}
            }
        }
    \end{center}
}


\newenvironment{alglist}{\begin{list}{}{\setlength{\leftmargin}{1.5cm}
\setlength{\rightmargin}{0cm}\setlength{\itemsep}{1ex}\setlength{\parsep}{1ex}}}{\end{list}}

\newcommand{\problem}[3]
{\fbox{\parbox{6in}{{\bf #1}\begin{itemize}\item{\bf Input:} {#2} \item{\bf 
Goal:} {#3}\end{itemize}}}}
\usepackage{latex-macros}
\usepackage{cleveref}
\usepackage{todonotes}
\usepackage{tikz,pgfplots}
\usepackage{pstricks-add}
\begin{document}

\allowdisplaybreaks
%\everymath{\displaystyle}

\homework{\today}{Problem Set 2}{Vasileios Charisopoulos}{vc333}%
{Math 6410 - Enumerative Combinatorics}

\section*{Problem 3}
Let us consider an arbitrary face lattice $\cL$ of rank $\mathrm{rk}(\cL) = 3$.
By~\cite[Theorem 2.7]{Zieg95}, we know that the face lattice of a polytope $P$
admits a rank function that satisfies $\mathrm{rk}(F) = \dim(F) + 1, F$ a face
of $P$. Since the rank of a graded lattice is equal to the rank of a maximal
chain, we deduce that $\mathrm{rk}(\cL(P)) = 3 \Leftrightarrow \dim(P) = 2$.

Now, we show that given any two face lattices of rank $3$, $\cL(P), \cL(P')$, 
with equinumerous levels, we can deduce that they are the same and so they
correspond to the same kind of polygon. The cases of $\hat{0}, \hat{1}$ are 
trivial: in both $\cL(P)$ and $\cL(P')$, they correspond to the empty set and 
the whole polygon respectively.

The first and second levels being equinumerous implies that the two polygons 
have the same number of vertices and edges.  Therefore, $\cL(P), \cL(P')$ both 
correspond to a $n$-gon in the plane, and all $n$-gons have the same face 
lattice (up to isomorphism): every vertex is incident to exactly $2$ edges and
every edge connects $2$ vertices, so we can pick an arbitrary starting vertex 
in $P$ and $P'$ and traverse the cycles 
\[
    v_1 \to v_2 \to \dots \to v_n \to v_1, \;
    v'_1 \to v'_2 \to \dots \to v'_n \to v'_1
\]
so the mapping $v_i \leftrightarrow v'_i$ is an isomorphism between their face
lattices, as edges are uniquely defined by $(v_i, v_j) \leftrightarrow (v'_i, 
v'_j)$ and hence $\cL(P) \simeq \cL(P')$.
\bibliographystyle{plain}
\bibliography{references}

\end{document}
