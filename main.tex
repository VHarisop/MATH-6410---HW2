\documentclass[11pt]{article}

% Required packages
\usepackage{bm,bbm}
\usepackage{amsmath,amssymb,amsthm,cancel}
\usepackage{fullpage, algorithm, algpseudocode}
\usepackage{minted, caption}

% Color references
\usepackage[
    colorlinks=true, citecolor=blue, linkcolor=blue]{hyperref}

% More shortcuts
\newcommand{\ve}{\varepsilon}
\newcommand{\dl}{\delta}

%\everymath{\displaystyle}

\newcommand{\homework}[5]{
	\noindent
    \begin{center}
    	\framebox{
        	\vbox{
            	% Course Title and Date
            	\hbox to \hsize { \textsc{ #5 } \hfill #1 }
                \vspace{4mm}
                % Title of handout/homework
                \hbox to \hsize { {\Large \hfill \textsc{#2} \hfill} }
                \vspace{2mm}
                % Name, Cornell NetID
                \hbox to \hsize {\hfill \textbf{\textsf{#3}} - \texttt{#4@cornell.edu} \hfill }
                \vspace{1mm}
            }
        }
    \end{center}
}


\newenvironment{alglist}{\begin{list}{}{\setlength{\leftmargin}{1.5cm}
\setlength{\rightmargin}{0cm}\setlength{\itemsep}{1ex}\setlength{\parsep}{1ex}}}{\end{list}}

\newcommand{\problem}[3]
{\fbox{\parbox{6in}{{\bf #1}\begin{itemize}\item{\bf Input:} {#2} \item{\bf 
Goal:} {#3}\end{itemize}}}}
\usepackage{latex-macros}
\usepackage{cleveref}
\usepackage{todonotes}
\usepackage{tikz,pgfplots}
\usepackage{pstricks-add}
\begin{document}

\allowdisplaybreaks
%\everymath{\displaystyle}

\homework{\today}{Problem Set 2}{Vasileios Charisopoulos}{vc333}%
{Math 6410 - Enumerative Combinatorics}

\section*{Problem 1}
Consider a distributive lattice $L$. By the fundamental theorem of finite
distributive lattices~\cite[Theorem 3.4.1]{Stan12}, we know that there exists
a unique (up to isomorphism) poset $P$ such that $L \simeq J(P)$.
Additionally, we know that the number of multichains of length $n$ in $J(P)$
that have the form
\[
    \hat{0} = t_0 \leq t_1 \dots \leq t_n = \hat{1}
\]
is equal to the number of order-preserving maps $\sigma : P \to \bm{n}$,
by~\cite[Proposition 3.5.1]{Stan12}. Therefore, since $Z(B_k, n)$ counts the
number of multichains of length $n - 2$ in $B_k$ - or, equivalently, the number
of multichains of length $ n $ that have the form
\[
    \hat{0} = t_0 \leq t_1 \dots \leq t_{n-1} \leq t_n = \hat{1},
\]
it suffices to find a lattice $L$ such that $B_k = J(L)$ for which the number
of order-preserving maps $\pi: L \to \bm{n}$ is easy to calculate. Hence our
claim:
\begin{claim}
    $B_k \simeq J(k \bm{1})$, where $k \bm{1} =
    \underbrace{\bm{1} + \dots + \bm{1}}_{k \text{ times}}$.
\end{claim}
\begin{proof}
    Every subset of $k \bm{1}$ is an order ideal, as any two elements of
    the disjoint union $k \bm{1}$ are incomparable. Since $J(P)$ is the poset
    of order ideals of $P$ ordered by inclusion and the order ideals of $k
    \bm{1}$ are exactly its subsets, we must have
    $J(k \bm{1}) \simeq B_k$.
\end{proof}
Now, since $k \bm{1}$ is a disjoint product of $1$'s, all maps from $k \bm{1}
\to \bm{n}$ are order preserving. The number of maps $X \to Y$ for discrete sets
$X, Y$ is given by $\abs{Y}^{\abs{X}}$, therefore we conclude that
\[
    Z(B_k, n) = \#\set{f \mmid f : k\bm{1} \to \bm{n}} = n^k,
\]
which agrees with the non-bijective proof in~\cite[Example 3.12.2]{Stan12}.

\section*{Problem 3}
Let us consider an arbitrary face lattice $\cL$ of rank $\mathrm{rk}(\cL) = 3$.
By~\cite[Theorem 2.7]{Zieg95}, we know that the face lattice of a polytope $P$
admits a rank function that satisfies $\mathrm{rk}(F) = \dim(F) + 1, F$ a face
of $P$. Since the rank of a graded lattice is equal to the rank of a maximal
chain, we deduce that $\mathrm{rk}(\cL(P)) = 3 \Leftrightarrow \dim(P) = 2$.

Now, we show that given any two face lattices of rank $3$, $\cL(P), \cL(P')$,
with equinumerous levels, we can deduce that they are the same and so they
correspond to the same kind of polygon. The cases of $\hat{0}, \hat{1}$ are
trivial: in both $\cL(P)$ and $\cL(P')$, they correspond to the empty set and
the whole polygon respectively.

The first and second levels being equinumerous implies that the two polygons
have the same number of vertices and edges.  Therefore, $\cL(P), \cL(P')$ both
correspond to a $n$-gon in the plane, and all $n$-gons have the same face
lattice (up to isomorphism): every vertex is incident to exactly $2$ edges and
every edge connects $2$ vertices, so we can pick an arbitrary starting vertex
in $P$ and $P'$ and traverse the cycles
\[
    v_1 \to v_2 \to \dots \to v_n \to v_1, \;
    v'_1 \to v'_2 \to \dots \to v'_n \to v'_1
\]
so the mapping $v_i \leftrightarrow v'_i$ is an isomorphism between their face
lattices, as edges are uniquely defined by $(v_i, v_j) \leftrightarrow (v'_i,
v'_j)$ and hence $\cL(P) \simeq \cL(P')$.

\section*{Problem 4}
First, recall that given a poset $(P, \leq)$, its order ideals are precisely
the sets $I \subset P$ such that
\[
    t \in I, s \leq t \Rightarrow s \in I
\]
The corresponding \textit{dual order ideals} are defined in a symmetric fashion:
a set $J \subset P$ is a dual order ideal if
\[
    t \in J, t \leq s \Rightarrow s \in J.
\]
Given a (finite) poset $(P, \leq)$ we can associate it with the \textit{order
polytope} $\cO(P)$, which is defined as follows: we denote the elements of $P$
by $x_1, \dots, x_n$, where $n$ is the size of $P$, and write
\begin{equation}
    \Rbb^n \supseteq \cO(P) \equiv \begin{cases}
        0 \leq z_i \leq 1, & \forall x_i \in P, \\
        z_i \leq z_j, & \text{if } x_i \leq x_j \text{ in P}
    \end{cases}
    \label{eq:order_poly_def}
\end{equation}
Notice that the definition in~\cref{eq:order_poly_def} is essentially the same
as the one given in~\cite{Stan86}, but we implicitly substitute $z_i = f(x_i)$.
We start with an auxiliary claim:

\begin{claim}
The vertices of $\cO(P)$ lie in the set $\set{0, 1}^n$.
\label{claim:0_1_vert}
\end{claim}
\begin{proof}
Consider some $\bm{z} \in \cO(P)$ with a coordinate $z_i$ such that $0 < z_i <
1$. Take $z_* \tdef \min\set{z_i \mmid 0 < z_i < 1}$, and generate the new
points $\bm{z}^+, \bm{z}^-$, with
\[	z^+_i = \begin{cases}
    z_i, & z_i \neq z_* \\
    z_* + \veps, & z_i = z_*
\end{cases}, \quad
z^-_i = \begin{cases}
    z_i, & z_i \neq z_* \\
    z_* - \veps, & z_i = z_*
\end{cases}.
\] It is easy to verify that $\bm{z}^+, \bm{z}^- \in \cO(P)$ and that
$\bm{z} = \frac{1}{2}\left( \bm{z}^+ + \bm{z}^- \right)$, that is $\bm{z}
\in \convhull\set{\bm{z}^+, \bm{z}^-}$, hence $\bm{z}$ cannot be a vertex.
\end{proof}
Given a dual order ideal $I$, consider its characteristic function,
\[
    \chi_I(x) = \begin{cases}
        1, & x \in I \\
        0, & \text{ otherwise}
    \end{cases}.
\]
Additionally, denote the set of all dual order ideals of $P$ by $\bar{J}(P)$.
We will prove the following:
\begin{claim}
    There exists a bijection between the characteristic functions of the dual
    order ideals of $P$ and the vertices of $\cO(P)$.
\end{claim}
\begin{proof}
    Consider a dual order ideal $I$, and take its characteristic function:
    \[
        \chi_I(x_i) = \begin{cases}
            1, & x_i \in I \\
            0, & x_i \notin I
        \end{cases}
    \]
    We can map $I$ to a vertex $\bm{v}$, by considering
    \[
        \bm{v}_i = \chi_I(x_i)
    \]
    It is immediately obvious that $\bm{v}_i \in \set{0, 1}^n$. Additionally,
    $\bm{v} \in \cO(P)$, as the following is satisfied: $x_j \geq x_i, x_i
    \in I \Rightarrow x_j \in I$ by the definition of the dual order ideal,
    and $\bm{v}_j = 1 \geq 1 = \bm{v}_i$, so $\bm{v} \in \cO(P)$. Therefore,
    the inclusion $\set{\chi_I \mmid I \in \bar{J}(P)} \subseteq
    \mathrm{vert}(\cO(P)) $ holds.

    Now, we will prove that $\mathrm{vert}(\cO(P)) \subseteq
    \set{\chi_I \mmid I \in \bar{J}(P)}$. Take a vertex $\bm{v} \in
    \cO(P)$ and a pair of its coordinates $v_j = 0, v_i = 1$. Then, we
    have 2 cases:
    \begin{itemize}
    \item $x_i \leq x_j$ in $P$: this cannot happen by the definition of $
    \cO(P)$, which would require $z_i \leq z_j, \; \forall \bm{z} \in \cO(P)
    \Rightarrow v_i \leq v_j$.
    \item $x_i \geq x_j$ or $x_i, x_j$ incomparable in $P$: in this case,
    $\bm{v}$ corresponds to an ideal $I$ whose characteristic satisfies
    $\chi_I(x_i) = 1, \chi_I(x_j) = 0$. Such an ideal exists since $x_j \ngeq
    x_i$, otherwise we would have $x_j \in I \Rightarrow \chi_I(x_j) = 1$.
    \end{itemize}
    Therefore we obtain $\mathrm{vert}(\cO(P)) = \set{\chi_I \mmid
    I \in \bar{J}(P)}$.
\end{proof}
\bibliographystyle{plain}
\bibliography{references}

\end{document}
